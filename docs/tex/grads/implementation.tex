% !TEX TS-program = pdflatex
% !TEX encoding = UTF-8 Unicode

% This is a simple template for a LaTeX document using the "article" class.
% See "book", "report", "letter" for other types of document.

\documentclass[10pt]{article} % use larger type; default would be 10pt
\usepackage[utf8]{inputenc} % set input encoding (not needed with XeLaTeX)
\usepackage[usenames,dvipsnames]{color}

%%% PAGE DIMENSIONS
\usepackage[top=.9in, bottom=1in, left=.5in, right=.5in]{geometry} % to change the page dimensions
\geometry{a4paper} % or letterpaper (US) or a5paper or....
% \geometry{margin=2in} % for example, change the margins to 2 inches all round
% \geometry{landscape} % set up the page for landscape
%   read geometry.pdf for detailed page layout information

\usepackage{graphicx} % support the \includegraphics command and options

% \usepackage[parfill]{parskip} % Activate to begin paragraphs with an empty line rather than an indent

\usepackage{booktabs} % for much better looking tables
\usepackage{array} % for better arrays (eg matrices) in maths
\usepackage{paralist} % very flexible & customisable lists (eg. enumerate/itemize, etc.)
\usepackage{verbatim} % adds environment for commenting out blocks of text & for better verbatim
\usepackage{subfig} % make it possible to include more than one captioned figure/table in a single float
\usepackage{verbatim}
\usepackage{amsmath}
\usepackage{algorithm}
\usepackage{algorithmic}
\usepackage{url}
\usepackage{longtable}
\usepackage{makeidx}
\usepackage{pdfpages}
\usepackage{wasysym}
\usepackage{blindtext}
\usepackage[inline]{enumitem}
\usepackage{draftwatermark}
\usepackage{xcolor}
\usepackage[backend=bibtex,style=ieee]{biblatex}
\usepackage[autostyle]{csquotes} 
\usepackage{epigraph}
\usepackage{authblk}

% Package for marking matrices 
\usepackage{tikz}
\usetikzlibrary{fit}
\tikzset{%
	highlight/.style={rectangle,rounded corners,fill=red!15,draw,
		fill opacity=0.5,thick,inner sep=0pt}
}
\newcommand{\tikzmark}[2]{\tikz[overlay,remember picture,
	baseline=(#1.base)] \node (#1) {#2};}

%\usepackage{fullpage}

%%% HEADERS & FOOTERS
\usepackage{lastpage}
\usepackage{fancyhdr} % This should be set AFTER setting up the page geometry
\pagestyle{fancy} % options: empty , plain , fancy
\renewcommand{\headrulewidth}{0pt} % customise the layout...
\lhead{}
\chead{\rightmark} 
\rhead{\thepage\ of \pageref{LastPage}}

\lfoot{}
\cfoot{{\color{Gray} \small https://github.com/abhijat01/autodiff.light \normalsize}}
\rfoot{}

%%% SECTION TITLE APPEARANCE
\usepackage{sectsty}
\allsectionsfont{\sffamily\mdseries\upshape} % (See the fntguide.pdf for font help)
% (This matches ConTeXt defaults)

%%% ToC (table of contents) APPEARANCE
\usepackage[nottoc,notlof,notlot]{tocbibind} % Put the bibliography in the ToC
\usepackage[titles,subfigure]{tocloft} % Alter the style of the Table of Contents
\renewcommand{\cftsecfont}{\rmfamily\mdseries\upshape}
\renewcommand{\cftsecpagefont}{\rmfamily\mdseries\upshape} % No bold!
\newcommand{\twopartdef}[4]
{
	\left\{
	\begin{array}{ll}
		#1 & \mbox{ } #2 \\
		#3 & \mbox{ } #4
	\end{array}
	\right.
}
\makeindex

\SetWatermarkText{Draft}
\SetWatermarkScale{4}
\SetWatermarkLightness{0.8}

\usepackage[normalem]{ulem}
\usepackage{lineno}
\usepackage[colorlinks=true,urlcolor=Gray]{hyperref}
\DeclareMathSizes{10}{8}{7}{5}


\title{Notes on gradient implementations in autodiff.light}
\author{Abhijat Vatsyayan}
\date{}
\begin{document}
\setcounter{secnumdepth}{5}
\maketitle
%\tableofcontents

\section{Convolution}
This is used to document the implementation of 2D convolution  in 
autodiff.light so that others can understand and improve the implementation. Let us start 
with the  input matrix X, Kernel w,  and the result of the convolution y with zero padding 
and stride of 1. 
\begin{equation*} 
\begin{aligned}
X &= 
	\begin{bmatrix}
	x_{11} & x_{12} & x_{13} & x_{14} \\
	x_{21} & x_{22} & x_{23} & x_{24} \\
	x_{31} & x_{32} & x_{33} & x_{34}
	\end{bmatrix}  
\\
w &= 
	\begin{bmatrix}
	w_{11} & w_{12} \\
	w_{21} & w_{22} 
	\end{bmatrix} 
\\
Y & = 
\begin{bmatrix}
	x_{11}w_{11}  + x_{12}w_{12}  + x_{21}w_{21}  + x_{22}w_{22} & 
	x_{12}w_{11}  + x_{13}w_{12}  + x_{22}w_{21}  + x_{23}w_{22} & 
	x_{13}w_{11}  + x_{14}w_{12}  + x_{23}w_{21}  + x_{24}w_{22} \\
	x_{21}w_{11}  + x_{22}w_{12}  + x_{31}w_{21}  + x_{32}w_{22} & 
	x_{22}w_{11}  + x_{23}w_{12}  + x_{32}w_{21}  + x_{33}w_{22} & 
	x_{23}w_{11}  + x_{24}w_{12}  + x_{33}w_{21}  + x_{34}w_{22} 
\end{bmatrix}  
\\
 & = 
 \begin{bmatrix}
	y_{11} & y_{12} & y_{13} \\
	y_{21} & y_{22} & y_{23} 
 \end{bmatrix}
\end{aligned} 
\end{equation*} 
You can imagine sliding ${\bf w}$ as a window sliding over ${\bf x}$. 

\subsection{Calculating gradient with respect to the input ${\bf x}$}
Let us look at the gradient w.r.t. a single input component $x_{11}$ 
\begin{equation*}
\begin{split}
\frac{\partial L}{\partial x_{11}} & = \frac{\partial L}{\partial y_{11}} \frac{\partial y_{11}}{\partial x_{11}} 
+ \frac{\partial L}{\partial y_{12}} \frac{\partial y_{12}}{\partial x_{11}} 
+ \frac{\partial L}{\partial y_{13}} \frac{\partial y_{13}}{\partial x_{11}} +\dots \\
  & = \frac{\partial L}{\partial y_{11}} w_{11} +0 \dots 
\end{split} 
\end{equation*}


$\frac{\partial L}{\partial x_{22}}$ is more interesting 
\begin{equation*}
\begin{split}
\frac{\partial L}{\partial x_{22}} & = 
\frac{\partial L}{\partial y_{11}} \frac{\partial y_{11}}{\partial x_{22}} +
\frac{\partial L}{\partial y_{12}} \frac{\partial y_{12}}{\partial x_{22}} + 
\frac{\partial L}{\partial y_{13}} \frac{\partial y_{13}}{\partial x_{22}} +  
\frac{\partial L}{\partial y_{21}} \frac{\partial y_{21}}{\partial x_{22}} +
\frac{\partial L}{\partial y_{22}} \frac{\partial y_{22}}{\partial x_{22}} + 
\frac{\partial L}{\partial y_{23}} \frac{\partial y_{23}}{\partial x_{22}} \\
 & = 
\frac{\partial L}{\partial y_{11}} w_{22} +
\frac{\partial L}{\partial y_{12}} w_{21} + 
\frac{\partial L}{\partial y_{13}} \times 0 +  
\frac{\partial L}{\partial y_{21}} w_{12} +
\frac{\partial L}{\partial y_{22}} w_{11} + 
\frac{\partial L}{\partial y_{23}} \times 0 \\
\end{split}
\end{equation*}



\section{Softmax} 
It is a  mapping 
\begin{equation}
	\begin{aligned}
		\begin{bmatrix}
			y_{11} \\
			y_{21} \\
			\vdots \\
			y_{m1}  
		\end{bmatrix} & \longrightarrow  
		\begin{bmatrix}
			u_{11} \\
			u_{21} \\
			\vdots \\
			u_{m1}
		\end{bmatrix}   \\	
	\end{aligned} \\
\end{equation}

Dropping second subscript for simplicity, the following is basic calculus without taking into account the 
exact function that maps $\vec{{\bf y}}$ to $\vec{{\bf u}}$
\begin{equation}
	\begin{aligned}
		\begin{bmatrix}
			y_{1} \\
			y_{2} \\
			\vdots \\
			y_{m}  
		\end{bmatrix} & \longrightarrow  
		\begin{bmatrix}
			u_{1} \\
			u_{2} \\
			\vdots \\
			u_{m}
		\end{bmatrix}   \\
		u_{1} & = \frac{e^{y_1}}{e^{y_1} + e^{y_2} + \dots + e^{y_m}}
	\end{aligned}
\end{equation}
Where $u_{1}  = \frac{e^{y_1}}{e^{y_1} + e^{y_2} + \dots + e^{y_m}} $ , and in general, 
$u_{i}  = \frac{e^{y_i}}{e^{y_1} + e^{y_2} + \dots + e^{y_i} + \dots + e^{y_m}} $
\subsection{Gradient calculations}
\begin{equation}
\begin{aligned}
\frac{\partial L}{\partial y_{1}} & = \frac{\partial L}{\partial u_{1}}\frac{\partial u_{1}}{\partial y_{1}} + 
									\frac{\partial L}{\partial u_{2}}\frac{\partial u_{2}}{\partial y_{1}} +
									\dots +  
									\frac{\partial L}{\partial u_{m}}\frac{\partial u_{m}}{\partial y_{1}}   \\
								 & = 		
								 \begin{bmatrix}
									 \frac{\partial u_{1}}{\partial y_{1}}  & \frac{\partial u_{2}}{\partial y_{1}}  &  \dots  & 
									 \frac{\partial u_{m}}{\partial y_{1}} \\
								 \end{bmatrix}	
								 \begin{bmatrix}
									 \frac{\partial L}{\partial u_{1}} \\
									 \frac{\partial L}{\partial u_{2}} \\
									 \vdots \\ 
									 \frac{\partial L}{\partial u_{m}}
								 \end{bmatrix} \\
\begin{bmatrix} 
\frac{\partial L}{\partial y_{1}} \\
\frac{\partial L}{\partial y_{2}} \\
\vdots \\
\frac{\partial L}{\partial y_{m}} 
\end{bmatrix} & = 	
\begin{bmatrix}
\frac{\partial u_{1}}{\partial y_{1}}  & \frac{\partial u_{2}}{\partial y_{1}}  &  \dots  & \frac{\partial u_{m}}{\partial y_{1}} \\
\frac{\partial u_{1}}{\partial y_{2}}  & \frac{\partial u_{2}}{\partial y_{2}}  &  \dots  & \frac{\partial u_{m}}{\partial y_{2}} \\
\vdots 	                               & \vdots                                 &         & \vdots \\ 
\frac{\partial u_{1}}{\partial y_{m}}  & \frac{\partial u_{2}}{\partial y_{m}}  &  \dots  & \frac{\partial u_{m}}{\partial y_{m}} 
\end{bmatrix} 	
\begin{bmatrix}
\frac{\partial L}{\partial u_{1}} \\
\frac{\partial L}{\partial u_{2}} \\
\vdots \\ 
\frac{\partial L}{\partial u_{m}}
\end{bmatrix}  \\
\frac{\partial u_i}{\partial y_j} & = \frac{\partial}{\partial y_j} (\frac{e^{y_i}}{e^{y_1}+\dots + e^{y_i}+\dots+e^{y_m}})\\
								  & = \twopartdef {u_i(1-u_i)} {i=j}{-u_i u_j}{i \neq j} \\  		  
\end{aligned}
\end{equation}



\end{document}