% !TEX TS-program = pdflatex
% !TEX encoding = UTF-8 Unicode

% This is a simple template for a LaTeX document using the "article" class.
% See "book", "report", "letter" for other types of document.

\documentclass[10pt]{article} % use larger type; default would be 10pt
\usepackage[utf8]{inputenc} % set input encoding (not needed with XeLaTeX)
\usepackage[usenames,dvipsnames]{color}

%%% PAGE DIMENSIONS
\usepackage[top=.9in, bottom=1in, left=.5in, right=.5in]{geometry} % to change the page dimensions
\geometry{a4paper} % or letterpaper (US) or a5paper or....
% \geometry{margin=2in} % for example, change the margins to 2 inches all round
% \geometry{landscape} % set up the page for landscape
%   read geometry.pdf for detailed page layout information

\usepackage{graphicx} % support the \includegraphics command and options

% \usepackage[parfill]{parskip} % Activate to begin paragraphs with an empty line rather than an indent

\usepackage{booktabs} % for much better looking tables
\usepackage{array} % for better arrays (eg matrices) in maths
\usepackage{paralist} % very flexible & customisable lists (eg. enumerate/itemize, etc.)
\usepackage{verbatim} % adds environment for commenting out blocks of text & for better verbatim
\usepackage{subfig} % make it possible to include more than one captioned figure/table in a single float
\usepackage{verbatim}
\usepackage{amsmath}
\usepackage{algorithm}
\usepackage{algorithmic}
\usepackage{url}
\usepackage{longtable}
\usepackage{makeidx}
\usepackage{pdfpages}
\usepackage{wasysym}
\usepackage{blindtext}
\usepackage[inline]{enumitem}
\usepackage{draftwatermark}
\usepackage{xcolor}
\usepackage[backend=bibtex,style=ieee]{biblatex}
\usepackage[autostyle]{csquotes} 
\usepackage{epigraph}
\usepackage{authblk}

%\usepackage{fullpage}

%%% HEADERS & FOOTERS
\usepackage{lastpage}
\usepackage{fancyhdr} % This should be set AFTER setting up the page geometry
\pagestyle{fancy} % options: empty , plain , fancy
\renewcommand{\headrulewidth}{0pt} % customise the layout...
\lhead{}
\chead{\rightmark} 
\rhead{\thepage\ of \pageref{LastPage}}

\lfoot{}
\cfoot{{\color{Gray} \small https://github.com/abhijat01/autodiff.light \normalsize}}
\rfoot{}

%%% SECTION TITLE APPEARANCE
\usepackage{sectsty}
\allsectionsfont{\sffamily\mdseries\upshape} % (See the fntguide.pdf for font help)
% (This matches ConTeXt defaults)

%%% ToC (table of contents) APPEARANCE
\usepackage[nottoc,notlof,notlot]{tocbibind} % Put the bibliography in the ToC
\usepackage[titles,subfigure]{tocloft} % Alter the style of the Table of Contents
\renewcommand{\cftsecfont}{\rmfamily\mdseries\upshape}
\renewcommand{\cftsecpagefont}{\rmfamily\mdseries\upshape} % No bold!

\makeindex

\SetWatermarkText{Draft}
\SetWatermarkScale{4}
\SetWatermarkLightness{0.8}

\usepackage[normalem]{ulem}
\usepackage{lineno}
\usepackage[colorlinks=true,urlcolor=Gray]{hyperref}
\DeclareMathSizes{10}{8}{7}{5}


\title{2D Convolution implementation in autodiff.light}
\author{Abhijat Vatsyayan}
\date{}
\begin{document}
\setcounter{secnumdepth}{5}
\maketitle
\tableofcontents
\section{Introduction}
This is used to document the implementation of 2D convolution implementation in 
autodiff.light so that others can understand and improve the implementation. Let us start 
with the  input matrix X, Kernel w,  and the result of the convolution y, 
\begin{gather*} 
\begin{split}
X &= 
\begin{bmatrix}
x_{11} & x_{12} & x_{13} & x_{14} \\
x_{21} & x_{22} & x_{23} & x_{24} \\
x_{31} & x_{32} & x_{33} & x_{34}
\end{bmatrix}  
\\
w &= \begin{bmatrix}
w_{11} & w_{12} \\
w_{21} & w_{22} 
\end{bmatrix} 
\\
y & = \begin{bmatrix}
x_{11}w_{11}  + x_{12}w_{12}  + x_{21}w_{21}  + x_{22}w_{22} 
& 
x_{12}w_{11}  + x_{13}w_{12}  + x_{22}w_{21}  + x_{23}w_{22} 
& 
x_{13}w_{11}  + x_{14}w_{12}  + x_{23}w_{21}  + x_{24}w_{22} \\
x_{21}w_{11}  + x_{22}w_{12}  + x_{31}w_{21}  + x_{32}w_{22} 
& 
x_{22}w_{11}  + x_{23}w_{12}  + x_{32}w_{21}  + x_{33}w_{22} 
& 
x_{23}w_{11}  + x_{24}w_{12}  + x_{33}w_{21}  + x_{34}w_{22} 
\end{bmatrix} 
\end{split} 
\end{gather*} 
Note that these matrices use $0$ based indexing - it is easier to convert this to code 
but even more importantly, latex code for these matrices was generating using {\em Sympy} 
which uses $0$ based 
indexing.




%\begin{figure}[!htb]
%	\begin{center}
%	\includegraphics[keepaspectratio=true,width=.7\linewidth]{diagrams/DesignExpTemplate.png}
%	\caption{Designing experiment templates}
%	\label{fig:exp-design}
%	\end{center}
%\end{figure} 

\end{document}
